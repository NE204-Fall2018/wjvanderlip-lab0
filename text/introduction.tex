%Introduction / motivation / background go here.
Peak fitting is a fundamental and essential exercise for conduting gamma ray spectrosopy.
The ability to take raw data from a detector and turn it in to the initial stage of usable
data is a task all individuals studying nuclear science should know and understand. We first begin to
understand the information contained in each particle by determining its energy.  The methods to make this
determination vary, but for this lab a HPGe detector with a 13-bit resolution MCA, producting 8192 bin spectra was used.
Through proper analysis and manipulation of this data, a wealth of information can be determined.
This lab faciliated the usage of modern data analysis techniques to take gamma ray energy data from
multiple sources and conduct an enrgy calibration.

\subsection*{Motivation}
\label{sec:motivation}

The purpose of this report has three key elements:

\begin{enumerate}
  \item To learn how to write lab reports using \LaTeX and the NE 204 template.
  \item To learn to use {\tt numpy} and {\tt scipy} to fit models to data.
  \item Te lean how to colborate with others in the field to acheive a common goal.
\end{enumerate}

Each one of these elements, will better prepare us to undertake for advanced experements in the future.
The ability to properly manage and store data, as well as to collaborate with our peers can not be overlooked.
